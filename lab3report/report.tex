\documentclass[a4paper]{article}
\usepackage{latexsym}
\usepackage[a4paper]{geometry}
\usepackage{color}
\usepackage{listings}
\usepackage[pdftex]{graphicx}
\usepackage{subfig}
\usepackage{float}

\definecolor{Blue}{rgb}{0,0,0.5}
\definecolor{Green}{rgb}{0,0.75,0.0}
\definecolor{LightGray}{rgb}{0.6,0.6,0.6}
\definecolor{DarkGray}{rgb}{0.3,0.3,0.3}
\lstset{language=Matlab,
   keywords={function,uint8,uint16,uint32,double,break,case,catch,continue,else,elseif,end,for,global,if,otherwise,persistent,return,switch,try,while},
   basicstyle=\ttfamily\small,
   breaklines=true,
   keywordstyle=\bfseries\color{Blue},
   commentstyle=\itshape\color{LightGray},
   stringstyle=\color{Green},
   numbers=left,
   numberstyle=\tiny\color{DarkGray},
   stepnumber=1,
   numbersep=10pt,
   backgroundcolor=\color{white},
   tabsize=2,
   showspaces=false,
   showstringspaces=false,
   captionpos=b}

%Boldface text for type writer font
\usepackage{bold-extra} %\DeclareFontShape{OT1}{cmtt}{bx}{n}{<5><6><7><8><9><10><10.95><12><14.4><17.28><20.74><24.88>cmttb10}{}

%Break words properly at the end of a line (which isn't sloppy...)
\sloppy

%Use command \exercise for each exercise
\newcounter{exerciseCount}
\setcounter{exerciseCount}{0}
\newcommand{\exercise}[1]{\addtocounter{exerciseCount}{1} \noindent \medskip {\large \textsf{\textbf{Exercise \arabic{exerciseCount} \--- #1}}} \par}
\renewcommand{\theenumi}{\textsf{\textbf{\alph{enumi}}}}

%Use command \code for code snippets
\newcommand{\code}[1]{\textnormal{\texttt{#1}}}



\title{\textsf{Image Processing \\ lab 3}}
\author{Klaas Kliffen \and Jan Kramer}
\date{\today}

\begin{document}
\maketitle

\exercise{1-D wavelet transforms}
\begin{enumerate}
\item

%Assuming the length of the input vector is a power of two,
%it is possible to iterate over the input vector applying the algoritm described by example 7.8.
%The current length is initialized to the length of the vector. For each step, this length is halved.
%Since the input vector of a step is split in half, a part with sums and a part with differences,
%the next step will only use the sums, as specified by the algoritm.
%First two vector are created by taking the odd and even element of the input vector until the current length of the scale.
%Two result vectors are calculated for the sums and the differences.
%The difference vector is positioned after the sum vector and written to the input vector.
%The values of the sum and difference vector needs normalization, which is a factor of the square root of 2.
The algorithm given in the assignment can be represented as a filter bank based on the Haar scaling and wavelet vectors.
Normally the whole input would be convolved with these vectors.
However since the two-point sums and differences are taken this is convolution is combined with downscaling.
Our implementation of a $j$-scale DWT is based on this algorithm by applying this ``filter bank'' $j$ times.
Otherwise its implementation is rather trivial.

\lstinputlisting{../lab3ex1/IPdwt.m}
\item 
For the inverse wavelet transform, the inital length is set to the end of the length of the last sum vector.
Then for each step the sum and difference vectors are retrieved from the input vector.
Two new component vectors are created for the values. These need to be scaled again to retrieve the right results.
The component vectors are then interleaved and replace their original values in the input vector.
This is repeated until the orignal scale of the transformation is reached.
\lstinputlisting{../lab3ex1/IPidwt.m}
\end{enumerate}

\newpage

\exercise{2-D wavelet transforms}
\begin{enumerate}
\item
According to the Section $7.5$ in the book the extension of 1D DWT to 2D is simple, because of the separable scaling and wavelet functions.
It also mentions that the 2D DWT can be computed by first doing a 1D DWT of the columns and then doing a 1D DWT of the rows.
Note that one could also calculate the DWT first for rows and then for columns.
Our implementation uses this fact.
In each iteration $j$ the algorithm of exercise 1 is applied to each row and then to each column to get the approximation, the horizontal detail, the vertical detail and the diagonal detail.

\lstinputlisting{../lab3ex2/IPdwt2.m}
\item
TODO: text
\lstinputlisting{../lab3ex2/IPdwt2scale.m}
\item
TODO: text
\lstinputlisting{../lab3ex2/runex2.m}

\begin{figure}[H]
\centering
\begin{tabular}{cc}
 \includegraphics[width=0.4\textwidth]{../lab3ex2/vase.png} & \includegraphics[width=0.4\textwidth]{../lab3ex2/scaled.png} \\
 Original image & 3 scale wavelet transform \\
\end{tabular}
\caption{3 scale wavelet transform of the orignal image}
\label{fig:wavelet}
\end{figure}

\item
According to Section $7.5$ the 2D inverse DWT can also be computed by using a 1D inverse DWT function.
So similarly on how our 2D DWT implementation is based on our 1D DWT implemention, the 2D inverse DWT is also based on the 1D DWT implementation.
Note however that the order of applying 1D DWT first to rows and then columns in the 2D DWT, has to be inverted in the 2D inverse DWT.
Hence our iplementation first applies a 1D inverse DWT to the columns and the one to the rows.
\lstinputlisting{../lab3ex2/IPidwt2.m}
\begin{figure}[H]
\centering
\begin{tabular}{cc}
 \includegraphics[width=0.4\textwidth]{../lab3ex2/vase.png} & \includegraphics[width=0.4\textwidth]{../lab3ex2/output.png} \\
 Original image & Restoration of the image form figure \ref{fig:wavelet} \\
\end{tabular}
\caption{3 scale wavelet transform restoration}
\label{fig:waveletback}
\end{figure}

\end{enumerate}

\exercise{Image Compression}
\begin{enumerate}
\item
First the wavelet transform is performed on the input image.
A threshold matrix consisting solely of the threshold value is used to find all pixels larger than the threshold in the transform.
The thresholded image is then retrieved by pointwise multiplication of the results of comparing the threshold matrix with the wavelet transform.
Since the approximation of the original matrix is also passed by the threshold, it needs to be reconstructed.
This is done by copying the values from the wavelet transform back to the thresholded image.
The compressed image is then retrieved by applying the inverse wavelet transform.

\noindent The root mean square error and the mean square signal to noise ratio are calculated from the given formulas.
For the compression ratio the the build-in function of entropy is used to calculate the entropy of the compressed and
original image.
%TODO: Jan, klopt dit een beetje of is dit een beetje vaag/kort door de bocht.
The entropy is a value for how complex an image is. Compressing the image lowers the complexity.
Dividing the original entropy by the compressed entropy yields a compression ratio.

\lstinputlisting{../lab3ex3/IPwaveletcompress.m}
\item
The quantitative properties for several different scales and thresholds can be seen in table \ref{tab:results}.
Globally the signal to noise ratio decreases and the errors and compression ratio increase while increasing scale and threshold.
Although for this image the increasing the scale past 9 was not possible, due to its size being a square of 512 pixels.
It would seem that increasing it further would not compress the image any further past a compression ratio of 37.
Increasing the threshold will yield in higher compression ratios. Although the signal to noise ratio decreases exponentiall, while
the errors increase almost linearly.
\begin{table}[H]
\centering
\begin{tabular}{r|r|r|r|r}
 \textbf{Scale} & \textbf{Threshold} & \textbf{$\epsilon_{rms}$} & \textbf{SNR} & \textbf{Compression ratio}\\
 \hline
 1 & 0.02 & 0.0086 & 1277.34 & 2.59:1 \\
 3 & 0.02 & 0.0157 &  385.47 & 17.42:1 \\
 3 & 0.05 & 0.0249 & 152.09  & 24.87:1 \\
 3 & 0.10 & 0.0322 & 90.33  & 29.14:1 \\
 5 & 0.02 & 0.0220 & 194.71  & 34.00:1 \\
 7 & 0.02 & 0.0272 & 127.44  & 36.40:1 \\
 7 & 0.05 & 0.0575 & 27.627  & 113.18:1 \\
 7 & 0.10 & 0.1035 & 7.837  & 460.20:1 \\
 9 & 0.02 & 0.0321 & 91.02   & 36.55:1 \\
\end{tabular}
\caption{Quantitative compression quality for different scales and threshold}
\label{tab:results}
\end{table}

\begin{figure}[H]
\centering
\begin{tabular}{cc}
 \includegraphics[width=0.4\textwidth]{../lab3ex3/tracy.png} & \includegraphics[width=0.4\textwidth]{../lab3ex3/l1t002.png} \\
 Original image & Scale: 1 Threshold: 0.02 \\
 \includegraphics[width=0.4\textwidth]{../lab3ex3/l5t002.png} & \includegraphics[width=0.4\textwidth]{../lab3ex3/l9t002.png}  \\
 Scale: 5 Threshold: 0.02 & Scale: 9 Threshold: 0.02 
\end{tabular}
\caption{Increasing wavelet compression scale on an image}
\label{fig:scaleinc}
\end{figure}

\begin{figure}[H]
\centering
\begin{tabular}{cc}
 \includegraphics[width=0.4\textwidth]{../lab3ex3/tracy.png} & \includegraphics[width=0.4\textwidth]{../lab3ex3/l3t002.png} \\
 Original image & Scale: 3 Threshold: 0.02 \\
 \includegraphics[width=0.4\textwidth]{../lab3ex3/l3t005.png} & \includegraphics[width=0.4\textwidth]{../lab3ex3/l3t010.png}  \\
 Scale: 3 Threshold: 0.05 & Scale: 3 Threshold: 0.10 
\end{tabular}
\caption{Increasing wavelet compression threshold on scale 3 on an image}
\label{fig:scale3}
\end{figure}

\begin{figure}[H]
\centering
\begin{tabular}{cc}
 \includegraphics[width=0.4\textwidth]{../lab3ex3/tracy.png} & \includegraphics[width=0.4\textwidth]{../lab3ex3/l7t002.png} \\
 Original image & Scale: 7 Threshold: 0.02 \\
 \includegraphics[width=0.4\textwidth]{../lab3ex3/l7t005.png} & \includegraphics[width=0.4\textwidth]{../lab3ex3/l7t010.png}  \\
 Scale: 7 Threshold: 0.05 & Scale: 7 Threshold: 0.10 
\end{tabular}
\caption{Increasing wavelet compression threshold on scale 7 on an image}
\label{fig:scale7}
\end{figure}


\end{enumerate}

\newpage
\section*{Task distribution}

\begin{table}[H]
\centering
\begin{tabular}{ccccc}
ex1 & design & implementation & answers questions & writing report \\
\hline
Klaas & 60\% & 90\% & n.a. & 75\% \\
\hline
Jan & 40\% & 10\% & n.a. & 25\% \\
\end{tabular}
\end{table}

\begin{table}[H]
\centering
\begin{tabular}{ccccc}
ex2 & design & implementation & answers questions & writing report \\
\hline
Klaas & 50\% & 50\% & 50\% & 50\% \\
\hline
Jan & 50\% & 50\% & 50\% & 50\% \\
\end{tabular}
\end{table}

\begin{table}[H]
\centering
\begin{tabular}{ccccc}
ex3 & design & implementation & answers questions & writing report \\
\hline
Klaas & 50\% & 75\% & 50\% & 75\% \\
\hline
Jan & 50\% & 25\% & 50\% & 25\% \\
\end{tabular}
\end{table} 



\end{document}
