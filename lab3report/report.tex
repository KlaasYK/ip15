\documentclass[a4paper]{article}
\usepackage{latexsym}
\usepackage[a4paper]{geometry}
\usepackage{color}
\usepackage{listings}
\usepackage[pdftex]{graphicx}
\usepackage{subfig}
\usepackage{float}

\definecolor{Blue}{rgb}{0,0,0.5}
\definecolor{Green}{rgb}{0,0.75,0.0}
\definecolor{LightGray}{rgb}{0.6,0.6,0.6}
\definecolor{DarkGray}{rgb}{0.3,0.3,0.3}
\lstset{language=Matlab,
   keywords={function,uint8,uint16,uint32,double,break,case,catch,continue,else,elseif,end,for,global,if,otherwise,persistent,return,switch,try,while},
   basicstyle=\ttfamily\small,
   breaklines=true,
   keywordstyle=\bfseries\color{Blue},
   commentstyle=\itshape\color{LightGray},
   stringstyle=\color{Green},
   numbers=left,
   numberstyle=\tiny\color{DarkGray},
   stepnumber=1,
   numbersep=10pt,
   backgroundcolor=\color{white},
   tabsize=2,
   showspaces=false,
   showstringspaces=false,
   captionpos=b}

%Boldface text for type writer font
\usepackage{bold-extra} %\DeclareFontShape{OT1}{cmtt}{bx}{n}{<5><6><7><8><9><10><10.95><12><14.4><17.28><20.74><24.88>cmttb10}{}

%Break words properly at the end of a line (which isn't sloppy...)
\sloppy

%Use command \exercise for each exercise
\newcounter{exerciseCount}
\setcounter{exerciseCount}{0}
\newcommand{\exercise}[1]{\addtocounter{exerciseCount}{1} \noindent \medskip {\large \textsf{\textbf{Exercise \arabic{exerciseCount} \--- #1}}} \par}
\renewcommand{\theenumi}{\textsf{\textbf{\alph{enumi}}}}

%Use command \code for code snippets
\newcommand{\code}[1]{\textnormal{\texttt{#1}}}



\title{\textsf{Image Processing \\ lab 3}}
\author{Klaas Kliffen \and Jan Kramer}
\date{\today}

\begin{document}
\maketitle

\exercise{1-D wavelet transforms}
\begin{enumerate}
\item
The algorithm given in the assignment can be represented as a filter bank based on the Haar scaling and wavelet vectors.
Normally the whole input would be convolved with these vectors.
However since the two-point sums and differences are taken this is convolution is combined with downscaling.
Our implementation of a $j$-scale DWT is based on this algorithm by applying this ``filter bank'' $j$ times.
Otherwise its implementation is rather trivial.
\lstinputlisting{../lab3ex1/IPdwt.m}
\item 
% TODO
\lstinputlisting{../lab3ex1/IPidwt.m}
\end{enumerate}

\newpage

\exercise{2-D wavelet transforms}
\begin{enumerate}
\item
According to the Section $7.5$ in the book the extension of 1D DWT to 2D is simple, because of the separable scaling and wavelet functions.
It also mentions that the 2D DWT can be computed by first doing a 1D DWT of the columns and then doing a 1D DWT of the rows.
Note that one could also calculate the DWT first for rows and then for columns.
Our implementation uses this fact.
In each iteration $j$ the algorithm of exercise 1 is applied to each row and then to each column to get the approximation, the horizontal detail, the vertical detail and the diagonal detail.
\lstinputlisting{../lab3ex2/IPdwt2.m}
\item
\lstinputlisting{../lab3ex2/IPdwt2scale.m}
\item
\lstinputlisting{../lab3ex2/runex2.m}
\item
According to Section $7.5$ the 2D inverse DWT can also be computed by using a 1D inverse DWT function.
So similarly on how our 2D DWT implementation is based on our 1D DWT implemention, the 2D inverse DWT is also based on the 1D DWT implementation.
Note however that the order of applying 1D DWT first to rows and then columns in the 2D DWT, has to be inverted in the 2D inverse DWT.
Hence our iplementation first applies a 1D inverse DWT to the columns and the one to the rows.
\lstinputlisting{../lab3ex2/IPidwt2.m}
\end{enumerate}

\exercise{Image Compression}
\begin{enumerate}
\item
\lstinputlisting{../lab3ex3/IPwaveletcompress.m}
\item
\end{enumerate}

\newpage
\section*{Task distribution}

\begin{table}[H]
\centering
\begin{tabular}{ccccc}
ex1 & design & implementation & answers questions & writing report \\
\hline
Klaas & 50\% & 100\% & 50\% & 50\% \\
\hline
Jan & 50\% & 0\% & 50\% & 50\% \\
\end{tabular}
\end{table}

\begin{table}[H]
\centering
\begin{tabular}{ccccc}
ex2 & design & implementation & answers questions & writing report \\
\hline
Klaas & 50\% & 50\% & 50\% & 50\% \\
\hline
Jan & 50\% & 50\% & 50\% & 50\% \\
\end{tabular}
\end{table}

\begin{table}[H]
\centering
\begin{tabular}{ccccc}
ex3 & design & implementation & answers questions & writing report \\
\hline
Klaas & 50\% & 100\% & 50\% & 50\% \\
\hline
Jan & 50\% & 0\% & 50\% & 50\% \\
\end{tabular}
\end{table} 



\end{document}
